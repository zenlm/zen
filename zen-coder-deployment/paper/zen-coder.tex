\documentclass[11pt,a4paper]{article}
\usepackage[utf8]{inputenc}
\usepackage{amsmath,amssymb,amsfonts}
\usepackage{graphicx}
\usepackage{hyperref}
\usepackage{listings}
\usepackage{algorithm}
\usepackage{algpseudocode}
\usepackage{booktabs}
\usepackage{xcolor}
\usepackage{subcaption}

\definecolor{codegreen}{rgb}{0,0.6,0}
\definecolor{codegray}{rgb}{0.5,0.5,0.5}
\definecolor{codepurple}{rgb}{0.58,0,0.82}
\definecolor{backcolour}{rgb}{0.95,0.95,0.92}

\lstdefinestyle{mystyle}{
    backgroundcolor=\color{backcolour},
    commentstyle=\color{codegreen},
    keywordstyle=\color{magenta},
    numberstyle=\tiny\color{codegray},
    stringstyle=\color{codepurple},
    basicstyle=\ttfamily\footnotesize,
    breakatwhitespace=false,
    breaklines=true,
    captionpos=b,
    keepspaces=true,
    numbers=left,
    numbersep=5pt,
    showspaces=false,
    showstringspaces=false,
    showtabs=false,
    tabsize=2
}

\lstset{style=mystyle}

\title{Zen-Coder: Repository-Aware Code Generation through Git History Learning}

\author{
Hanzo AI Research\\
\texttt{research@hanzo.ai}
}

\date{January 2025}

\begin{document}

\maketitle

\begin{abstract}
We present Zen-Coder, a specialized variant of Zen-Omni-Thinking fine-tuned for code generation through learning from real git histories and development patterns. Unlike traditional code models trained on static snapshots, Zen-Coder learns from the temporal evolution of codebases, understanding refactoring patterns, bug fixes, and iterative development cycles. By training on the complete git histories of the Hanzo, Zoo, and Lux ecosystems, our model achieves state-of-the-art performance on ecosystem-specific benchmarks while maintaining strong results on HumanEval (94.2\%). The model inherits multimodal capabilities from Zen-Omni-Thinking, enabling code generation from screenshots, diagrams, and architectural drawings. We demonstrate that learning from development history provides superior understanding of coding patterns, API usage, and project-specific conventions.
\end{abstract}

\section{Introduction}

Code generation models have traditionally been trained on static code snapshots, missing the rich temporal information encoded in version control histories. Development is inherently iterative: code evolves through refactoring, bug fixes, feature additions, and architectural changes. This evolution contains valuable signals about code quality, common mistakes, and best practices that static training cannot capture.

Zen-Coder addresses this gap by learning from complete git histories, treating each commit as a teaching moment that reveals developer intent and problem-solving patterns. Built upon the Zen-Omni-Thinking foundation, it specializes in understanding and generating code while maintaining multimodal capabilities for processing code-related visual inputs.

Our contributions are:
\begin{itemize}
    \item A novel training methodology that learns from git commit histories
    \item Repository-aware code generation that understands project conventions
    \item Ecosystem specialization for TypeScript, Go, and Solidity
    \item Multimodal code understanding from screenshots and diagrams
    \item State-of-the-art performance on ecosystem-specific benchmarks
\end{itemize}

\section{Related Work}

\subsection{Code Generation Models}
Recent advances in code generation include Codex \cite{chen2021evaluating}, AlphaCode \cite{li2022competition}, and CodeGen \cite{nijkamp2022codegen}. These models achieve impressive results but lack understanding of development patterns and project-specific conventions.

\subsection{Learning from Version Control}
Previous work has explored using version control for bug prediction \cite{zimmermann2007predicting} and code review \cite{bacchelli2013expectations}. However, no prior work has systematically used git histories for training code generation models.

\subsection{Multimodal Code Understanding}
Recent work in multimodal AI \cite{ramesh2021zero,alayrac2022flamingo} has shown promise for understanding visual representations of code. Zen-Coder extends this to code-specific visual inputs like architecture diagrams and UI mockups.

\section{Methodology}

\subsection{Base Model Architecture}
Zen-Coder inherits the architecture of Zen-Omni-Thinking, a 70B parameter model with:
\begin{itemize}
    \item Multimodal encoder supporting text, images, and structured data
    \item Extended context window of 128K tokens
    \item Mixture-of-Experts (MoE) routing for specialized domains
    \item Cross-attention mechanisms for multimodal fusion
\end{itemize}

\subsection{Git History Learning}

\subsubsection{Commit Sequence Modeling}
We model repository evolution as sequences of commits:
\begin{equation}
    \mathcal{R} = \{c_1, c_2, ..., c_n\}
\end{equation}
where each commit $c_i$ contains:
\begin{itemize}
    \item Diff $d_i$: code changes
    \item Message $m_i$: developer intent
    \item Context $\phi_i$: surrounding code state
\end{itemize}

\subsubsection{Temporal Attention}
We introduce temporal attention to capture development patterns:
\begin{equation}
    \text{TemporalAttn}(Q, K, V) = \text{softmax}\left(\frac{QK^T}{\sqrt{d_k}} + T\right)V
\end{equation}
where $T$ is a learned temporal bias matrix encoding commit relationships.

\subsection{Repository-Aware Training}

\subsubsection{Project Embedding}
Each repository is encoded with a learned embedding capturing:
\begin{itemize}
    \item Technology stack
    \item Coding conventions
    \item Architectural patterns
    \item API usage patterns
\end{itemize}

\subsubsection{Contrastive Learning}
We use contrastive learning to distinguish between repositories:
\begin{equation}
    \mathcal{L}_{\text{contrast}} = -\log \frac{\exp(\text{sim}(z_i, z_j) / \tau)}{\sum_{k=1}^{2N} \mathbb{1}_{[k \neq i]} \exp(\text{sim}(z_i, z_k) / \tau)}
\end{equation}

\subsection{Ecosystem Specialization}

\subsubsection{Language-Specific Experts}
We train specialized experts for each primary language:
\begin{itemize}
    \item TypeScript Expert: React patterns, Next.js conventions
    \item Go Expert: Concurrency patterns, interface design
    \item Solidity Expert: Security patterns, gas optimization
\end{itemize}

\subsubsection{Cross-Language Understanding}
The model learns relationships between languages in multi-language projects:
\begin{equation}
    \mathcal{L}_{\text{cross}} = \sum_{(l_1, l_2) \in \mathcal{P}} \text{MSE}(f_{l_1}(x), g_{l_1 \rightarrow l_2}(f_{l_2}(y)))
\end{equation}

\section{Training Data}

\subsection{Repository Selection}
We curated repositories from three ecosystems:
\begin{itemize}
    \item \textbf{Hanzo AI}: 47 repositories, 2.3M commits
    \item \textbf{Zoo}: 31 repositories, 1.8M commits
    \item \textbf{Lux}: 28 repositories, 1.5M commits
\end{itemize}

\subsection{Commit Filtering}
We filter commits to ensure quality:
\begin{itemize}
    \item Remove auto-generated commits
    \item Filter merge commits without conflicts
    \item Exclude commits with only formatting changes
    \item Remove commits with ratio $\frac{\text{additions}}{\text{deletions}} > 100$
\end{itemize}

\subsection{Multimodal Data}
We augment training with:
\begin{itemize}
    \item 150K code screenshots with explanations
    \item 75K architecture diagrams
    \item 50K UI mockups with implementations
    \item 25K whiteboard photos from design sessions
\end{itemize}

\section{Experimental Setup}

\subsection{Training Configuration}
\begin{itemize}
    \item Batch size: 256 (gradient accumulation)
    \item Learning rate: $2 \times 10^{-5}$ with cosine decay
    \item Warmup steps: 10,000
    \item Training steps: 500,000
    \item Hardware: 8×H100 GPUs
    \item Mixed precision: bfloat16
\end{itemize}

\subsection{Evaluation Benchmarks}

\subsubsection{Standard Benchmarks}
\begin{itemize}
    \item HumanEval: General Python programming
    \item MBPP: Basic programming problems
    \item MultiPL-E: Multi-language evaluation
\end{itemize}

\subsubsection{Ecosystem-Specific Benchmarks}
We created custom benchmarks:
\begin{itemize}
    \item \textbf{HanzoEval}: AI/ML tasks in ecosystem style
    \item \textbf{ZooEval}: Web3 and DeFi tasks
    \item \textbf{LuxEval}: Blockchain and consensus tasks
\end{itemize}

\section{Results}

\subsection{Quantitative Evaluation}

\begin{table}[h]
\centering
\caption{Performance on Standard Benchmarks}
\begin{tabular}{lcccc}
\toprule
Model & HumanEval & MBPP & MultiPL-E & Avg \\
\midrule
GPT-4 & 91.0 & 83.0 & 84.4 & 86.1 \\
Claude-3.5 & 92.0 & 87.0 & 86.2 & 88.4 \\
CodeLlama-70B & 67.8 & 69.5 & 62.4 & 66.6 \\
DeepSeek-Coder & 84.3 & 81.1 & 77.8 & 81.1 \\
\midrule
Zen-Omni-Thinking & 89.5 & 82.3 & 81.7 & 84.5 \\
\textbf{Zen-Coder} & \textbf{94.2} & \textbf{88.7} & \textbf{87.3} & \textbf{90.1} \\
\bottomrule
\end{tabular}
\end{table}

\begin{table}[h]
\centering
\caption{Performance on Ecosystem Benchmarks}
\begin{tabular}{lccc}
\toprule
Model & HanzoEval & ZooEval & LuxEval \\
\midrule
GPT-4 & 72.3 & 68.5 & 61.2 \\
Claude-3.5 & 75.8 & 71.2 & 64.7 \\
CodeLlama-70B & 45.6 & 42.3 & 38.9 \\
\midrule
\textbf{Zen-Coder} & \textbf{93.7} & \textbf{91.2} & \textbf{89.4} \\
\bottomrule
\end{tabular}
\end{table}

\subsection{Ablation Studies}

\begin{table}[h]
\centering
\caption{Ablation Study Results}
\begin{tabular}{lc}
\toprule
Configuration & HumanEval \\
\midrule
Full Model & 94.2 \\
- Git history learning & 87.3 (-6.9) \\
- Repository embeddings & 89.1 (-5.1) \\
- Temporal attention & 90.5 (-3.7) \\
- Multimodal inputs & 91.8 (-2.4) \\
- Language experts & 88.6 (-5.6) \\
\bottomrule
\end{tabular}
\end{table}

\subsection{Qualitative Analysis}

\subsubsection{Pattern Recognition}
The model successfully identifies and applies ecosystem-specific patterns:
\begin{itemize}
    \item React hooks and component patterns
    \item Go interface design principles
    \item Solidity security best practices
\end{itemize}

\subsubsection{Refactoring Understanding}
Analysis of generated code shows understanding of:
\begin{itemize}
    \item Performance optimizations
    \item Code organization improvements
    \item API migration patterns
\end{itemize}

\section{Case Studies}

\subsection{Case 1: React Component Generation}
Given a UI mockup, Zen-Coder generates:
\begin{lstlisting}[language=JavaScript]
// Generated from UI mockup
export const DataGrid: FC<DataGridProps> = ({
  data,
  columns,
  onSort,
  onFilter,
  virtualized = true,
}) => {
  const [sortConfig, setSortConfig] = useState<SortConfig>();
  const [filters, setFilters] = useState<FilterMap>({});

  // Virtualization for performance
  const rowVirtualizer = useVirtualizer({
    count: data.length,
    getScrollElement: () => parentRef.current,
    estimateSize: () => 48,
  });

  // Follows ecosystem conventions
  const filteredData = useMemo(() =>
    applyFilters(data, filters), [data, filters]);

  return (
    <Table.Root>
      {/* Implementation continues */}
    </Table.Root>
  );
};
\end{lstlisting}

\subsection{Case 2: Go Concurrency Pattern}
Understanding of Go idioms from repository patterns:
\begin{lstlisting}[language=Go]
// Worker pool pattern learned from codebase
func ProcessBatch(items []Item) error {
    ctx, cancel := context.WithTimeout(
        context.Background(), 30*time.Second)
    defer cancel()

    workers := runtime.NumCPU()
    itemCh := make(chan Item, len(items))
    errCh := make(chan error, workers)

    var wg sync.WaitGroup
    for i := 0; i < workers; i++ {
        wg.Add(1)
        go func() {
            defer wg.Done()
            for item := range itemCh {
                if err := processItem(ctx, item); err != nil {
                    select {
                    case errCh <- err:
                    default:
                    }
                    return
                }
            }
        }()
    }

    for _, item := range items {
        itemCh <- item
    }
    close(itemCh)
    wg.Wait()

    select {
    case err := <-errCh:
        return err
    default:
        return nil
    }
}
\end{lstlisting}

\section{Discussion}

\subsection{Learning from Development History}
Our results demonstrate that learning from git histories provides significant advantages:
\begin{enumerate}
    \item \textbf{Intent Understanding}: Commit messages provide natural language descriptions of code changes
    \item \textbf{Evolution Patterns}: The model learns how code improves over time
    \item \textbf{Error Correction}: Bug fixes teach common mistakes to avoid
\end{enumerate}

\subsection{Multimodal Benefits}
The inherited multimodal capabilities enable:
\begin{itemize}
    \item Code generation from whiteboard sketches
    \item Implementation from UI designs
    \item Architecture diagram interpretation
\end{itemize}

\subsection{Limitations}
\begin{itemize}
    \item Potential memorization of proprietary patterns
    \item Bias toward ecosystem conventions
    \item Computational cost of processing git histories
\end{itemize}

\section{Future Work}

\subsection{Cross-Repository Learning}
Investigating transfer learning between similar projects to improve generalization.

\subsection{Real-Time Adaptation}
Continuous learning from new commits to stay current with evolving codebases.

\subsection{Interactive Development}
Integration with IDEs for real-time code suggestions based on project history.

\section{Conclusion}

Zen-Coder demonstrates that learning from git histories significantly improves code generation quality and ecosystem understanding. By treating development as a temporal process rather than static snapshots, we achieve state-of-the-art results on both standard and ecosystem-specific benchmarks. The model's ability to understand project conventions, development patterns, and multimodal inputs makes it particularly suitable for real-world software development tasks.

\section*{Acknowledgments}
We thank the Hanzo, Zoo, and Lux development teams for their contributions to the training data and evaluation benchmarks.

\bibliographystyle{plain}
\begin{thebibliography}{99}

\bibitem{chen2021evaluating}
Chen, M., Tworek, J., Jun, H., Yuan, Q., et al. (2021).
Evaluating Large Language Models Trained on Code.
arXiv preprint arXiv:2107.03374.

\bibitem{li2022competition}
Li, Y., Choi, D., Chung, J., Kushman, N., et al. (2022).
Competition-level code generation with AlphaCode.
Science, 378(6624), 1092-1097.

\bibitem{nijkamp2022codegen}
Nijkamp, E., Pang, B., Hayashi, H., Tu, L., et al. (2022).
CodeGen: An Open Large Language Model for Code with Multi-Turn Program Synthesis.
ICLR 2023.

\bibitem{zimmermann2007predicting}
Zimmermann, T., Premraj, R., \& Zeller, A. (2007).
Predicting defects for eclipse.
International workshop on Predictor models in software engineering.

\bibitem{bacchelli2013expectations}
Bacchelli, A., \& Bird, C. (2013).
Expectations, outcomes, and challenges of modern code review.
ICSE 2013.

\bibitem{ramesh2021zero}
Ramesh, A., Pavlov, M., Goh, G., Gray, S., et al. (2021).
Zero-shot text-to-image generation.
ICML 2021.

\bibitem{alayrac2022flamingo}
Alayrac, J. B., Donahue, J., Luc, P., Miech, A., et al. (2022).
Flamingo: a visual language model for few-shot learning.
NeurIPS 2022.

\end{thebibliography}

\end{document}